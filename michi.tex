\documentclass[10pt]{article}
\usepackage[margin=0.8in,noheadfoot]{geometry}

\newcommand{\name}[1]{\begin{center}\Large{\textsc{#1}}\end{center}}
\newcommand{\HRule}{\noindent\rule{\linewidth}{0.1mm}}
\newcommand{\header}[1]{\vspace{0.4cm}\noindent\textsc{\large{#1}}\vspace{-0.2cm}\newline\HRule}
\newcommand{\subheader}[2]{\noindent \textbf{#1}\hspace{\stretch{1}}#2}
\pagestyle{empty}

\renewcommand{\name}[2]{\noindent \textsc{\LARGE{#1}}\hspace{\stretch{1}}#2\vspace{-0.2cm}}
\begin{document}

\name{Michi Mutsuzaki}{\texttt{Email:michi@cs.stanford.edu}}

\header{Professional Experience}

\subheader{Research Engineer, Yahoo! Inc.}{July 2011 - Present}
\begin{list}{\labelitemi}{\leftmargin=1em}
\item \textbf{Walnut} - Walnut is an elastic object store that aims to efficiently
support various workloads including storing large blobs, small key-value look up,
and HDFS-style block access. I've been involved in design discussions, defining APIs,
prototyping and benchmarking. 

\item \textbf{YCSB} - YCSB is an open source benchmark tool for cloud storage
systems. As the current maintainer, I'm responsible for adding new features, 
fixing bugs, reviewing and applying patches, and answering questions from the 
community.

\item \textbf{MapKeeper} - MapKeeper is an open source, Thrift-based key value
store that supports various storage engines, including MySQL, Berkeley DB, and 
LevelDB. I developed MapKeeper to evaluate different storage technologies for
Yahoo's cloud platform.
\end{list}
\vspace{0.2cm}

\subheader{Software Engineer, Yahoo! Inc.}{July 2007 - July 2011}
\begin{list}{\labelitemi}{\leftmargin=1em}
\item \textbf{PNUTS} - PNUTS is a geographically replicated NoSQL database. I
worked on various aspects of the system including HTTP proxy server, client
library, and evaluation of various storage engines.

\item \textbf{ZooKeeper} - ZooKeeper is an open source distributed contribution
system. My responsibilities as a committer included reviewing and applying patches,
fixing bugs, and running performance tests.
\end{list}

\vspace{0.2cm}

\subheader{Research Assistant, InfoLab, Stanford University}{June 2006 - July 2007}

\noindent
\begin{list}{\labelitemi}{\leftmargin=1em}
\item \textbf{Trio} - Trio is a database system that incorporate uncertainty
and lineage. I implemented the first prototype, including the core database
engine and a web interface to the system using PostgreSQL and Python.
\end{list}

\vspace{0.2cm}

\subheader{Tutor/Grader, UC Santa Cruz}{April 2004 - June 2005}

\noindent
\begin{list}{\labelitemi}{\leftmargin=1em}
\item Tutored courses in data structures, abstract data types, probability
theory, and communication systems. Held weekly discussion sections and office
hours, prepared solutions to and graded assignments.
\end{list}
\vspace{0.2cm}

\subheader{Research Assistant, UC Santa Cruz}{June 2004 - December 2004}

\noindent
\begin{list}{\labelitemi}{\leftmargin=1em}
\item Worked on a project on image processing with Prof. Roberto Manduchi. 
Implemented a target detection algorithm used to localize textual and barcode
information in images in C++ and a graphic user interface.
\end{list}
%\vspace{0.2cm}

\header{Computer Skills} 

\noindent
C, C++, Java, Python, MySQL, Berkeley DB, Berkeley DB Java Edition, LevelDB,
Thrift, ZooKeeper, Apache Traffic Server, YCSB.

\header{Education}
\subheader{M.S. in Computer Science, Stanford University. GPA: 3.7/4.0}{June 2007}

\subheader{B.S. in Computer Science, University of California, Santa Cruz. GPA: 3.97/4.0}{June 2005} 

\header{Other Skills}
Fluent in Japanese. 

\header{Publications}
\noindent
Mutsuzaki, Michi; Theobald, Martin; de Keijzer, Ander; Widom, Jennifer; 
Agrawal, Parag; Benjelloun, Omar; Das Sarma, Anish; Murthy, Raghotham; 
Sugihara, Tomoe. Trio-One: Layering Uncertainty and Lineage on a 
Conventional DBMS, Proc. of CIDR conference (system demonstration)
\newline

\noindent
J. Coughlan, R. Manduchi, M. Mutsuzaki, H. Shen, "Rapid and Robust
Algorithms to Detect Colour Targets", 10th Congress of the Colour 
Association, AIC, Granada, Spain, May 2005.

\end{document}
